\documentclass[10pt,a4paper]{article}
\usepackage[utf8]{inputenc}
\usepackage[T1]{fontenc}
\usepackage{amsmath}
\usepackage{amsfonts}
\usepackage{amssymb}
\usepackage{graphicx}
\usepackage[french]{babel}
\usepackage{geometry}
\usepackage{hyperref}
\usepackage{enumitem}
\usepackage{titlesec} % Ajouté pour contrôler l'espacement des titres

% --- CONFIGURATION DE LA PAGE ---
\geometry{a4paper, top=1.3cm, bottom=1.3cm, left=1.8cm, right=1.8cm}
\pagestyle{empty}
\raggedbottom

% --- CONFIGURATION DES HYPERLIENS ---
\hypersetup{
    colorlinks=true,
    linkcolor=black,
    urlcolor=blue,
    pdftitle={CV Youssef AIT ELOURF},
    pdfauthor={Youssef AIT ELOURF},
}

% --- AJUSTEMENT DE L'ESPACE APRÈS LES TITRES DE SECTION ---
\titlespacing*{\section}{0pt}{0pt}{4pt}

% --- COMMANDES PERSONNALISÉES ---
\newcommand{\sectionbreak}{\vspace{4pt}\hrule\vspace{5pt}}

% --- DÉBUT DU DOCUMENT ---
\begin{document}

% ====================
% EN-TÊTE
% ====================
\begin{center}
    {\Huge \textbf{Youssef AIT ELOURF}}
    \vspace{2pt}
    
    youssef.aitelourf@gmail.com $|$ +1 (581) 672-2103 $|$ \href{https://www.linkedin.com/in/youssef-ait-elourf-223316355/}{linkedin.com/in/youssef-ait-elourf}
\end{center}

% ====================
% PROFIL
% ====================
\section*{Profil}
\noindent
Étudiant en dernière année de cycle ingénieur Data \& IA à l'ECE Paris et en double diplomation à l'UQAR (Maîtrise en informatique IA et Machine Learning). Passionné par l'IA appliquée et les systèmes de données à grande échelle. Recherche un stage à partir de mai 2026 à Montréal/Québec en tant que Machine Learning Engineer, Data Scientist ou Data Engineer, avec un intérêt particulier pour les systèmes multi-agents, le NLP et l'IA générative appliquée aux domaines industriel, santé et finance.
\sectionbreak

% ====================
% EXPÉRIENCES PROFESSIONNELLES
% ====================
\section*{Expériences professionnelles}
\noindent
\textbf{Ingénieur en Machine Learning et Intelligence artificielle - Temps partiel} \hfill Août 2025 -- Nov. 2025 \\
\textit{Eddmon et Le Kompa} \hfill \textit{À distance (Canada) - Entreprise basée à Paris, France}
\begin{itemize}[leftmargin=*, label=\textbullet, topsep=1pt, itemsep=0pt, partopsep=0pt, parsep=0pt]
    \item Contribution active à la stratégie IA de l'entreprise (agents multi-domaines, NLP, IA générative).
    \item Optimisation et maintenance continue des modèles en production, réduisant les coûts cloud de ~15 \%.
    \item Développement de nouveaux projets IA parallèlement à mes études, générant un gain estimé de 4 à 6 h/semaine pour les équipes métiers.
\end{itemize}
\vspace{6pt} % Espace pour aérer

\noindent
\textbf{Ingénieur en Machine Learning et Intelligence artificielle - Stagiaire} \hfill Avril 2025 -- Août 2025 \\
\textit{Eddmon} \hfill \textit{Paris, France}
\begin{itemize}[leftmargin=*, label=\textbullet, topsep=1pt, itemsep=0pt, partopsep=0pt, parsep=0pt]
    \item Développement et déploiement d'agents IA intégrés au CRM et aux outils métiers, avec un taux d'adoption > 80 \% par les équipes.
    \item Mise en production d'outils adoptés par les équipes RH, Sales et CSM, réduisant le temps de traitement de certaines tâches de plusieurs minutes à quelques secondes.
    \item Projets : transcription et analyse d'appels (comptes-rendus instantanés, gain de 100 \% de temps de saisie), génération automatique de fiches clients (3–5 min → quelques secondes), réponses SMS/mails (délai de réponse divisé par 5), algorithme de matching pour demandes spécifiques (1 h → quelques secondes).
    \item \textbf{Outils:} Python, Hugging Face, LangChain, SQL, Docker, Google Cloud Platform, DigitalOcean.
\end{itemize}
\vspace{6pt} % Espace pour aérer

\noindent
\textbf{Architecte réseaux et cybersécurité - Stagiaire} \hfill Juin 2023 -- Août 2023 \\
\textit{ACG Cybersecurity} \hfill \textit{Paris, France}
\begin{itemize}[leftmargin=*, label=\textbullet, topsep=1pt, itemsep=0pt, partopsep=0pt, parsep=0pt]
    \item Conception d'architectures réseaux sécurisées (segmentation, firewalls, VPN, IDS/IPS), contribuant à réduire le risque d'incident critique de ~20 \%.
    \item Audit des infrastructures existantes et recommandations d'amélioration, mises en œuvre sur 3 projets clients.
    \item \textbf{Outils:} Cisco Packet Tracer, Wireshark, Nessus, protocoles VPN/IPSec/SSL.
\end{itemize}
\sectionbreak

% ====================
% FORMATION
% ====================
\section*{Formation}
\noindent
\textbf{Maîtrise en informatique -- IA et Machine Learning} \hfill 2025 - 2026 \\
\textit{Université du Québec à Rimouski (UQAR)} \hfill \textit{Rimouski, Canada}
\vspace{6pt} % Espace pour aérer

\noindent
\textbf{Diplôme d'ingénieur d'état -- Data et Intelligence Artificielle} \hfill 2022 -- 2026 \\
\textit{École centrale d'électronique de Paris (ECE Paris)} \hfill \textit{Paris, France}
\vspace{6pt} % Espace pour aérer

\noindent
\textbf{Classes préparatoires MPSI/PSI} \hfill 2020 -- 2022 \\
\textit{Lycée Franklin D. Roosevelt} \hfill \textit{Reims, France}
\sectionbreak

% ====================
% COMPÉTENCES TECHNIQUES
% ====================
\section*{Compétences techniques}
\begin{itemize}[leftmargin=*, label=\textbullet, topsep=1pt, itemsep=0pt, partopsep=0pt, parsep=0pt]
    \item \textbf{Langages:} Python, R, Java, SQL, Scala
    \item \textbf{IA, LLMs \& ML:} PyTorch, TensorFlow, Scikit-learn, Keras, LangChain, Hugging Face, Pandas, NumPy, XGBoost, Transformers, Vector Databases (e.g., Pinecone)
    \item \textbf{Big Data:} Apache Spark, Hadoop, Koalas
    \item \textbf{Bases de données:} MySQL, PostgreSQL, MongoDB, NoSQL
    \item \textbf{Cloud \& DevOps:} Azure (certifié), Amazon Web Services (AWS), Google Cloud Platform (GCP), DigitalOcean, Docker, Kubernetes, MLOps
\end{itemize}
\sectionbreak

% ====================
% LANGUES ET CERTIFICATIONS
% ====================
\section*{Langues et Certifications}
\begin{itemize}[leftmargin=*, label=\textbullet, topsep=1pt, itemsep=0pt, partopsep=0pt, parsep=0pt]
    \item \textbf{Langues:} Français (langue maternelle et DALF C1), Anglais (C1, TOEIC 955), Arabe (langue maternelle)
    \item \textbf{Certifications:} Azure AZ-900, Azure DP-100, MOOC Gestion de projets, Python Data Scientist
\end{itemize}

\end{document}